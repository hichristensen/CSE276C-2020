\documentclass[10pt]{beamer}

\mode<presentation>
{
  \usetheme[height=1.25cm]{Madrid}
  \setbeamertemplate{navigation symbols}{}
  \setbeamercolor{alerted text}{fg=illini}
}
\usebackgroundtemplate{\includegraphics[width=\paperwidth,height=\paperheight]{uc-background}}

\usepackage[english]{babel}
\usepackage{epsfig,subfigure,bm}
\usepackage{multimedia}
\usepackage{psfrag}
\usepackage{animate}

%%%%%% Begin of my macros and options

\setbeamertemplate{section in toc shaded}[default][55]
\setbeamertemplate{subsection in toc shaded}[default][55]
\setbeamercolor{block title}{fg=white,bg=illini}
\setbeamercolor{block body}{fg=black,bg=mygrey}

\setbeamercolor{emphprimary}{fg=CBlue}
\setbeamercolor{emphsecondary}{fg=illini}
\setbeamercolor{emphtertiary}{fg=mygreen}
\definecolor{darkForestGreen}{rgb}{.1,1,.1}
\definecolor{veryLightGray}{rgb}{.9,.9,.9}
\definecolor{greenApple}{rgb}{.3,.9,.3}

\setbeamercolor{frametitle}{bg=CBlue}   
\setbeamercolor{title}{bg=CBlue}

\usepackage{amsmath,amssymb,amsxtra,amsthm}
\usepackage{algorithm,algorithmic}
\usepackage{natbib}
\usepackage{bibentry}
\usepackage{xspace}
\usepackage{changepage}

\pdfmapfile{+sansmathaccent.map}

\definecolor{myblue}{rgb}{.2,.2,.7}
\definecolor{myred}{rgb}{.7,.2,.2}
\definecolor{mygreen}{rgb}{.2,.7,.2}
\definecolor{mygrey}{rgb}{0.9,0.9,0.9}
\definecolor{CBlue}{cmyk}{1,0.25,0,0}
\definecolor{illini}{rgb}{0.98,0.4,0.05}
\definecolor{black}{cmyk}{0,0,0,1}

\newcommand{\myemph}[1]{{\usebeamercolor[fg]{emphprimary}
    \textbf{#1}}}
\newcommand{\myemphalt}[1]{{\usebeamercolor[fg]{emphsecondary}
    \textbf{#1}}}

\graphicspath{{figs/}}

\title[Math for Robotics] % (optional, use only with long paper titles)
{CSE276C - Linear Systems of Equations}

\author[H.~I. Christensen] % (optional, use only with lots of authors)
{Henrik I.~Christensen}
% - Give the names in the same order as the appear in the paper.  -
% Use the \inst{?} command only if the authors have different
% affiliation.

\institute[UCSD] % (optional, but mostly needed)
{
  \begin{minipage}[c]{.2\textwidth}
    \includegraphics[width=.65\linewidth]{ucsealnew}%
  \end{minipage}%
  \begin{minipage}[c]{.6\textwidth}
    \small
%%    \begin{center}
      Computer Science and Engineering\\
      University of California, San Diego\\
      \myemph{\url{http://cri.ucsd.edu}}\\          
%%    \end{center}

  \end{minipage}
%%  \vspace*{1ex}
}
%% - Use the \inst command only if there are several affiliations.
%% - Keep it simple, no one is interested in your street address.

\bigskip

\date[Oct 2020]% (optional, should be abbreviation of conference name)
{\small%
  October 2020}

\begin{document}
  
\nobibliography{/Users/hic/Dropbox/bibliography/bib-file}
\bibliographystyle{plain}

\begin{frame}[plain]
  \titlepage
\end{frame}

\begin{frame}
  \frametitle{Outline}
  \begin{itemize}
  \item Linear Systems of Equations
  \item Solution Techniques - Gauss Jordan
  \item Matrix Decomposition
  \item Matrix Factorization
  \item Singular Value Decomposition
  \item Rank and sensitivity
  \end{itemize}
\end{frame}

\begin{frame}
  \frametitle{Linear Systems of Equations}
  \begin{itemize}
  \item One of the most basic tasks is solve for a set of unknowns
    \[
      \begin{array}{ccc}
        a_{00} x_0 + a_{01} x_1 + a_{02} x_2 + \ldots + a_{0n-1} x_{n-1} & = & b_0\\
        a_{10} x_0 + a_{11} x_1 + a_{12} x_2 + \ldots + a_{1n-1} x_{n-1} & = & b_1\\
        \vdots && \\
        a_{m-10} x_0 + a_{m-11} x_1 + a_{m-12} x_2 + \ldots + a_{m-1n-1} x_{n-1} & = & b_{m-1}\\
      \end{array}
    \]
    \pause
  \item which we can rewrite
    \[
      \mathbf{A} \vec{x} = \vec{b}
    \]
    where
    \[
      \mathbf{A} = \left(
        \begin{array}{ccccc}
          a_{00} & a_{01} & a_{01} & \cdots & a_{0n-1} \\
          a_{10} & a_{11} & a_{11} & \cdots & a_{1n-1} \\
          & &  \vdots & & \\
          a_{m-10} & a_{m-11} & a_{m-11} & \cdots & a_{m-1n-1} \\
        \end{array}\right)
      \vspace{1cm}
      \vec{b} = \left(
        \begin{array}{c}
          b_0 \\ b_1 \\ b_2 \\ \vdots \\ b_{m-1} \\
        \end{array}\right)
    \]    
  \end{itemize}
\end{frame}

\begin{frame}
  \frametitle{Matrix Properties}
  \begin{itemize}
  \item Given an $m \times n$ matrix A we define
    \begin{itemize}
    \item Column space - Linear combination of columns
    \item Row space - Linear combination of row
    \end{itemize}
  \item We can consider A a mapping:
    \[
      A: R^n \rightarrow R^m
    \]
    \[
      \left(
        \begin{array}{c}
          x_0 \\ x_1 \\ \vdots \\ x_{n-1} \\
        \end{array}
      \right) \rightarrow
      \left(
        \begin{array}{c}
          b_0 \\ b_1 \\ \vdots \\ b_{m-1} \\
        \end{array}
      \right) =
      \mathbf{A}
      \left(
        \begin{array}{c}
          x_0 \\ x_1 \\ \vdots \\ x_{n-1} \\
        \end{array}
      \right)
    \]
  \item Column space of A is vector subspace of $R^m$ that image
    vectors under A
  \end{itemize}
\end{frame}

\begin{frame}
  \frametitle{Null Space}
  \begin{itemize}
  \item We define the null-space: set of vectors $x \in R^n$ where
    \[
      A x = 0
    \]
  \item The row space and the null space are complementary
    \[
      n = dim(row~space) + dim(null~space)
    \]
  \end{itemize}
\end{frame}

\begin{frame}
  \frametitle{Questions}
  \centerline{\Huge Questions}
\end{frame}

\begin{frame}
  \frametitle{Gauss-Jordan Elimination}
  \begin{itemize}
  \item How can we solve the equation system? 
  \item The standard form
    \[
      \mathbf{A} \vec{x} = \vec{b} ~~~ \rightarrow ~~~ \mathbf{U} \vec{x}' = \vec{b}'
    \]
  \end{itemize}
\end{frame}
\end{document}
%%% Local Variables:
%%% mode: latex
%%% TeX-master: t
%%% End:
