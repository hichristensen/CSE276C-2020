\documentclass[10pt]{beamer}

\mode<presentation>
{
  \usetheme[height=1.25cm]{Madrid}
  \setbeamertemplate{navigation symbols}{}
  \setbeamercolor{alerted text}{fg=illini}
}
\usebackgroundtemplate{\includegraphics[width=\paperwidth,height=\paperheight]{uc-background}}

\usepackage[english]{babel}
\usepackage{epsfig,subfigure,bm}
\usepackage{multimedia}
\usepackage{psfrag}
\usepackage{animate}

%%%%%% Begin of my macros and options

\setbeamertemplate{section in toc shaded}[default][55]
\setbeamertemplate{subsection in toc shaded}[default][55]
\setbeamercolor{block title}{fg=white,bg=illini}
\setbeamercolor{block body}{fg=black,bg=mygrey}

\setbeamercolor{emphprimary}{fg=CBlue}
\setbeamercolor{emphsecondary}{fg=illini}
\setbeamercolor{emphtertiary}{fg=mygreen}
\definecolor{darkForestGreen}{rgb}{.1,1,.1}
\definecolor{veryLightGray}{rgb}{.9,.9,.9}
\definecolor{greenApple}{rgb}{.3,.9,.3}

\setbeamercolor{frametitle}{bg=CBlue}   
\setbeamercolor{title}{bg=CBlue}

\usepackage{amsmath,amssymb,amsxtra,amsthm}
\usepackage{algorithm,algorithmic}
\usepackage{natbib}
\usepackage{bibentry}
\usepackage{xspace}
\usepackage{changepage}

\pdfmapfile{+sansmathaccent.map}

\definecolor{myblue}{rgb}{.2,.2,.7}
\definecolor{myred}{rgb}{.7,.2,.2}
\definecolor{mygreen}{rgb}{.2,.7,.2}
\definecolor{mygrey}{rgb}{0.9,0.9,0.9}
\definecolor{CBlue}{cmyk}{1,0.25,0,0}
\definecolor{illini}{rgb}{0.98,0.4,0.05}
\definecolor{black}{cmyk}{0,0,0,1}

\newcommand{\myemph}[1]{{\usebeamercolor[fg]{emphprimary}
    \textbf{#1}}}
\newcommand{\myemphalt}[1]{{\usebeamercolor[fg]{emphsecondary}
    \textbf{#1}}}

\graphicspath{{figs/}}

\title[Math for Robotics] % (optional, use only with long paper titles)
{CSE276C - Roots of Polynomials}

\author[H.~I. Christensen] % (optional, use only with lots of authors)
{Henrik I.~Christensen}
% - Give the names in the same order as the appear in the paper.  -
% Use the \inst{?} command only if the authors have different
% affiliation.

\AtBeginSection[]
{
   \begin{frame}
       \frametitle{Outline}
       \tableofcontents[currentsection]
   \end{frame}
}

\institute[UCSD] % (optional, but mostly needed)
{
  \begin{minipage}[c]{.2\textwidth}
    \includegraphics[width=.65\linewidth]{ucsealnew}%
  \end{minipage}%
  \begin{minipage}[c]{.6\textwidth}
    \small
%%    \begin{center}
      Computer Science and Engineering\\
      University of California, San Diego\\
      \myemph{\url{http://cri.ucsd.edu}}\\          
%%    \end{center}

  \end{minipage}
%%  \vspace*{1ex}
}
%% - Use the \inst command only if there are several affiliations.
%% - Keep it simple, no one is interested in your street address.

\bigskip

\date[Oct 2020]% (optional, should be abbreviation of conference name)
{\small%
  October 2020}

\begin{document}
  
\nobibliography{/Users/hic/Dropbox/bibliography/bib-file}
\bibliographystyle{plain}

\begin{frame}[plain]
  \titlepage
\end{frame}

\section{Introduction}

\begin{frame}
  \frametitle{Introduction}
  \begin{itemize}
  \item Last time we looked at direct search for roots
  \item Bracketing was the way to limit the search domain
  \item Brent's method was a simple strategy to do search
  \item What if we have a polynomial?
    \begin{enumerate}
    \item Can we find the roots?
    \item Can we simplify the polynomial?
    \end{enumerate}
  \item Lets explore this
  \end{itemize}
\end{frame}

\section{Roots of Low Order Polynomials}

\begin{frame}
  \frametitle{Low order polynomials}
  \begin{itemize}
  \item We have closed form solutions to roots of polynomials up to degree 4
  \item Quadratics
    \[
      ax^2 + bx +c = 0, \mbox{~~~~} a\neq0
    \]
    has two roots
    \[
      x = \frac{-b \pm \sqrt{b^2 - 4ac}}{2a}
    \]
    we have real unique, dual or imaginary solutions
  \end{itemize}
\end{frame}

\begin{frame}
  \frametitle{Cubics}
  \begin{itemize}
  \item The cubic equation
    \[
      x^3 + px^2 + qx +r =0
    \] can be reduced using substitution
    \[
      x = y - \frac{p}{3}
    \] to the form
    \[
      y^3 +  a y + b = 0
    \] where
    \[
      \begin{array}{rcl}
        a &= & \frac{1}{3} (3q - p^2)\\
        b &= & \frac{1}{27} (2p^3 - 9 pq + 27r)\\
      \end{array}
    \] the condensed form has 3 roots
    \[
      \begin{array}{rcl}
        y_1& = & A+B\\
        y_2& = & -\frac{1}{2}(A+B) + \frac{i \sqrt{3}}{2} (A-B)\\
        y_3& = & -\frac{1}{2}(A+B) - \frac{i \sqrt{3}}{2} (A-B)\\
      \end{array}
    \] where
    \[
      \begin{array}{rclcrcl}
        A & = & \sqrt[3]{- \frac{b}{2} + \sqrt{\frac{b^2}{4} + \frac{a^3}{27}}} & \mbox {~~~} &
        B & = & \sqrt[3]{- \frac{b}{2} - \sqrt{\frac{b^2}{4} + \frac{a^3}{27}}}\\                                                                                                        
      \end{array}
    \]
  \end{itemize}
\end{frame}

\begin{frame}
  \frametitle{Cubic (cont)}
  \begin{itemize}
  \item We have three cases:
    \begin{enumerate}
    \item $\frac{b^2}{4} + \frac{a^3}{27} > 0$: one real root and two conjugate roots
    \item $\frac{b^2}{4} + \frac{a^3}{27} = 0$: three real roots of which at least two are equal
    \item $\frac{b^2}{4} + \frac{a^3}{27} < 0$: three real roots and unequal roots
    \end{enumerate}
  \end{itemize}
\end{frame}

\begin{frame}
  \frametitle{Quartics}
  \begin{itemize}
  \item 
  \end{itemize}
\end{frame}
\section{Root Counting}

\section{Bounds on Roots}

\section{Deflation}

\section{Newton's Method}

\section{M{\"u}ller's Method}

\section{Summary}


\end{document}

%%% Local Variables:
%%% mode: latex
%%% TeX-master: t
%%% End:
