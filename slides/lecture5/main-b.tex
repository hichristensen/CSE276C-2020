\documentclass[10pt]{beamer}

\mode<presentation>
{
  \usetheme[height=1.25cm]{Madrid}
  \setbeamertemplate{navigation symbols}{}
  \setbeamercolor{alerted text}{fg=illini}
}
\usebackgroundtemplate{\includegraphics[width=\paperwidth,height=\paperheight]{uc-background}}

\usepackage[english]{babel}
\usepackage{epsfig,subfigure,bm}
\usepackage{multimedia}
\usepackage{psfrag}
\usepackage{animate}

%%%%%% Begin of my macros and options

\setbeamertemplate{section in toc shaded}[default][55]
\setbeamertemplate{subsection in toc shaded}[default][55]
\setbeamercolor{block title}{fg=white,bg=illini}
\setbeamercolor{block body}{fg=black,bg=mygrey}

\setbeamercolor{emphprimary}{fg=CBlue}
\setbeamercolor{emphsecondary}{fg=illini}
\setbeamercolor{emphtertiary}{fg=mygreen}
\definecolor{darkForestGreen}{rgb}{.1,1,.1}
\definecolor{veryLightGray}{rgb}{.9,.9,.9}
\definecolor{greenApple}{rgb}{.3,.9,.3}

\setbeamercolor{frametitle}{bg=CBlue}   
\setbeamercolor{title}{bg=CBlue}

\usepackage{amsmath,amssymb,amsxtra,amsthm}
\usepackage{algorithm,algorithmic}
\usepackage{natbib}
\usepackage{bibentry}
\usepackage{xspace}
\usepackage{changepage}

\pdfmapfile{+sansmathaccent.map}

\definecolor{myblue}{rgb}{.2,.2,.7}
\definecolor{myred}{rgb}{.7,.2,.2}
\definecolor{mygreen}{rgb}{.2,.7,.2}
\definecolor{mygrey}{rgb}{0.9,0.9,0.9}
\definecolor{CBlue}{cmyk}{1,0.25,0,0}
\definecolor{illini}{rgb}{0.98,0.4,0.05}
\definecolor{black}{cmyk}{0,0,0,1}

\newcommand{\myemph}[1]{{\usebeamercolor[fg]{emphprimary}
    \textbf{#1}}}
\newcommand{\myemphalt}[1]{{\usebeamercolor[fg]{emphsecondary}
    \textbf{#1}}}

\graphicspath{{figs/}}

\title[Math for Robotics] % (optional, use only with long paper titles)
{CSE276C - Roots of Polynomials}

\author[H.~I. Christensen] % (optional, use only with lots of authors)
{Henrik I.~Christensen}
% - Give the names in the same order as the appear in the paper.  -
% Use the \inst{?} command only if the authors have different
% affiliation.

\AtBeginSection[]
{
   \begin{frame}
       \frametitle{Outline}
       \tableofcontents[currentsection]
   \end{frame}
}

\institute[UCSD] % (optional, but mostly needed)
{
  \begin{minipage}[c]{.2\textwidth}
    \includegraphics[width=.65\linewidth]{ucsealnew}%
  \end{minipage}%
  \begin{minipage}[c]{.6\textwidth}
    \small
%%    \begin{center}
      Computer Science and Engineering\\
      University of California, San Diego\\
      \myemph{\url{http://cri.ucsd.edu}}\\          
%%    \end{center}

  \end{minipage}
%%  \vspace*{1ex}
}
%% - Use the \inst command only if there are several affiliations.
%% - Keep it simple, no one is interested in your street address.

\bigskip

\date[Oct 2020]% (optional, should be abbreviation of conference name)
{\small%
  October 2020}

\begin{document}
  
\nobibliography{/Users/hic/Dropbox/bibliography/bib-file}
\bibliographystyle{plain}

\begin{frame}[plain]
  \titlepage
\end{frame}

\section{Introduction}

\begin{frame}
  \frametitle{Introduction}
  \begin{itemize}
  \item Last time we looked at direct search for roots
  \item Bracketing was the way to limit the search domain
  \item Brent's method was a simple strategy to do search
  \item What if we have a polynomial?
    \begin{enumerate}
    \item Can we find the roots?
    \item Can we simplify the polynomial?
    \end{enumerate}
  \item Lets explore this
  \end{itemize}
\end{frame}

\section{Roots of Low Order Polynomials}

\begin{frame}
  \frametitle{Low order polynomials}
  \begin{itemize}
  \item We have closed form solutions to roots of polynomials up to degree 4
  \item Quadratics
    \[
      ax^2 + bx +c = 0, \mbox{~~~~} a\neq0
    \]
    has two roots
    \[
      x = \frac{-b \pm \sqrt{b^2 - 4ac}}{2a}
    \]
    we have real unique, dual or imaginary solutions
  \end{itemize}
\end{frame}

\begin{frame}
  \frametitle{Cubics}
  \begin{itemize}
  \item The cubic equation
    \[
      x^3 + px^2 + qx +r =0
    \] can be reduced using substitution
    \[
      x = y - \frac{p}{3}
    \] to the form
    \[
      y^3 +  a y + b = 0
    \] where
    \[
      \begin{array}{rcl}
        a &= & \frac{1}{3} (3q - p^2)\\
        b &= & \frac{1}{27} (2p^3 - 9 pq + 27r)\\
      \end{array}
    \] the condensed form has 3 roots
    \[
      \begin{array}{rcl}
        y_1& = & A+B\\
        y_2& = & -\frac{1}{2}(A+B) + \frac{i \sqrt{3}}{2} (A-B)\\
        y_3& = & -\frac{1}{2}(A+B) - \frac{i \sqrt{3}}{2} (A-B)\\
      \end{array}
    \] where
    \[
      \begin{array}{rclcrcl}
        A & = & \sqrt[3]{- \frac{b}{2} + \sqrt{\frac{b^2}{4} + \frac{a^3}{27}}} & \mbox {~~~} &
        B & = & \sqrt[3]{- \frac{b}{2} - \sqrt{\frac{b^2}{4} + \frac{a^3}{27}}}\\                                                                                                        
      \end{array}
    \]
  \end{itemize}
\end{frame}

\begin{frame}
  \frametitle{Cubic (cont)}
  \begin{itemize}
  \item We have three cases:
    \begin{enumerate}
    \item $\frac{b^2}{4} + \frac{a^3}{27} > 0$: one real root and two conjugate roots
    \item $\frac{b^2}{4} + \frac{a^3}{27} = 0$: three real roots of which at least two are equal
    \item $\frac{b^2}{4} + \frac{a^3}{27} < 0$: three real roots and unequal roots
    \end{enumerate}
  \end{itemize}
\end{frame}

\begin{frame}
  \frametitle{Quartics}
  \begin{itemize}
  \item For the equation
    \[
      x^4 + p x^3 + q x^2 + r x + s =0
    \] we can apply a similar trick
    \[
      x = y - \frac{p}{4}
    \]
    to get
    \[
      y^4 + a y^2 + b y + c = 0
    \]
    where
    \[
      \begin{array}{rcl}
        a& = & q - \frac{3p^2}{8}\\
        b& = & r + \frac{p^3}{8} - \frac{pq}{2}\\
        c& = & s - \frac{4p^4}{256} + \frac{p^2 q}{16} - \frac{pr}{4}\\
      \end{array}
    \]
  \end{itemize}
\end{frame}
\begin{frame}
  \frametitle{Quartics (cont.)}
  \begin{itemize}
  \item The reduced equation can be factorized into
    \[
      z^3 - q z^2 + (pr-4s) z + (4sq - r^2 - p^2s) = 0
    \]
    if we can estimate $z_1$ of the above cubic then
    \[
      \begin{array}{rcl}
        x_1 &= & -\frac{p}{4} + \frac{1}{2} (R+D) \\
        x_2 &= & -\frac{p}{4} + \frac{1}{2} (R-D) \\
        x_3 &= & -\frac{p}{4} - \frac{1}{2} (R+E) \\
        x_4 &= & -\frac{p}{4} - \frac{1}{2} (R-D) \\
      \end{array}
    \] where
    \[
      \begin{array}{rcl}  
        R &=& \sqrt{\frac{1}{4} p^2 - q + z_1} \\
        D &=& \sqrt{\frac{3}{4} p^2 - R^2 - 2Q + \frac{1}{4}(4pq - 8r - p^3) R^{-1}}\\
        E &=& \sqrt{\frac{3}{4} p^2 - R^2 - 2Q - \frac{1}{4}(4pq - 8r - p^3) R^{-1}}\\
      \end{array}
    \]
  \end{itemize}
\end{frame}


\section{Root Counting}

\begin{frame}
  \frametitle{Root Counting}
  \begin{itemize}
  \item Consider a polynomial of degree n:
    \[
      p(x) - a_n x^n + a_{n-1} x^{n-1} + \ldots + a_1 x + a_0
    \]
  \item if $a_i$ are real the roots are real or complex conjugate pairs. 
  \item p(x) has n roots
  \item Descartes rules of sign
    \begin{itemize}
    \item ``The number of positive real zeroes in a polynomial
      function p(x) is the same or less than by an even numbers as the
      number of changes in the sign of the coefficients. The number of
      negative real zeroes of the p(x) is the same as the number of
      changes in sign of the coefficients of the terms of p(-x) or
      less than this by an even number''
    \end{itemize}
  \item Consider
    \[
      p(x) = x^5 + 4 x^4 - 3 x^2 + x -6
    \]
  \item So it must have 3 or 1 postive root and
  \item and it must have 2 or 0 negative roots
  \end{itemize}
\end{frame}

\begin{frame}
  \frametitle{Sturms theorem}
  \begin{itemize}
  \item We can derive a sequence of polynomials
  \item Let f(x) be a polynomial. Denote the original $f_0(x)$ and the
    derivative $f'(x) = f_1(x)$. Consider
    \[
      \begin{array}{rcl}
        f_0(x)      &=& q_1(x) f_1(x) - f_2(x) \\
        f_1(x)      &=& q_2(x) f_2(x) - f_3(x) \\
                 &\vdots& \\
        f__{k-2}(x) &=& q_{k-1}(x) f_{k-1}(x) - f_k(x) \\
        f__{k-1}(x) &=& q_k(x) f_k(x) \\
      \end{array}
    \]
  \item The theorem
    \begin{itemize}
    \item ``The number of distinct real zeros of a polynomial f(x)
      with real coefficients in (a, b) is equal to the excess of the
      number of changes of sign in the sequence $f_0(a), \ldots ,
      f_{k−1}(a)$, $f_k(a)$ over the number of changes of sign in the
      sequence $f_0(b), \ldots , f_{k−1}(b), f_k(b)$.''
    \end{itemize}
  \end{itemize}
\end{frame}

\begin{frame}
  \frametitle{Sturm - example}
  \begin{itemize}
  \item Consider the polynomial
    \[
      x^5 + 5x^4 - 20 x^2 - 10x + 2 =0
    \]
    The Sturm functions are then
    \[
      \begin{array}{rcl}
        f_0(x) & = &  x^5 + 5x^4 - 20 x^2 - 10x + 2\\
        f_1(x) & = &  x^4 + 4 x^3 - 8x - 2\\
        f_2(x) & = &  x^3 + 3 x^2 - 1\\
        f_3(x) & = &  3 x^2 + 7 x +1 \\
        f_4(x) & = &  17x + 11\\
        f(5(x) & = &  1\\
      \end{array}
    \]
  \end{itemize}
\end{frame}

\begin{frame}
  \frametitle{Sturm example (cont)}
  \centerline{\includegraphics[width=10cm]{sturm-figure}}
  \begin{itemize}
  \item So roots between (-4, -3), (-3, -2), (-1,0), (0,1) and (1,2)
  \end{itemize}
\end{frame}


\section{Deflation}

\begin{frame}
  \frametitle{Deflation}
  \begin{itemize}
  \item Once you have a root r you can deflate a polynomial
    \[
      p(x) = (x-r) q(x)
    \]
  \item As the degree decreases the complexity of root finding is simplified. 
  \item One can use Horner's scheme
    \[
      p(x) = b_0 + (x-r)(b_n x^{n-1} + \ldots + b_2 x + b_1)
    \]
    as r is a root $b_0 = 0$ so
    \[
      q(x) = b_n x^{n-1} + \ldots + b_2 x + b_1
    \]
  \end{itemize}
\end{frame}

\section{Newton's Method}

\begin{frame}
  \frametitle{Newton's Method}
  \begin{itemize}
  \item Remember we can do root search/refinement
    \[
      x_{k+1} = x_k - \frac{p(x_k)}{p'(x)}
    \] we know that
    \[
      p(x) = p(t) + (x-t) q(x)
    \]
    So p'(t) = q(t) or
    \[
      q(x) = \frac{p(x)}{x-t}
    \]
  \item If p(x) has double roots it could be a challenge
  \end{itemize}
\end{frame}

\section{M{\"u}ller's Method}

\begin{frame}
  \frametitle{M\"ullers Method}
  \begin{itemize}
  \item Newton's Method is local and sensitive to seed guess    
  \item M\"ullers method is more global
  \item Based on a quadratic interpolation
  \item Assume you have three estimates of the root: $x_{k-2}, x_{k-1}, x_k$
  \item Interpolation polynomial
    \[
      p(x) = f(x_k)
      + f[x_{k-1}, x_k](x-x_k)
      + f[x_{k-2}, x_{k-1}, x_k](x-x_k)(x-x_{k-1})
    \]
  \item Using the equality
    \[
      (x-x_k)(x-x_{k-1}) = (x-x_k)^2 + (x-x_k)(x_k-x_{k-1})
    \] we get
    \[
      p(x) = f(x_k) + b (x-x_k) + a (x-x_k)^2
    \]
    which we can solve for p(x) = 0
  \end{itemize}
\end{frame}

\section{Summary}

\begin{frame}
  \frametitle{Summary}
  \begin{itemize}
  \item Frequently using a polynomial refactorization is more stable
  \item A way to compress data into a semantic form
  \item For lower order polynomials we have closed for solutions
  \item We can use Descartes rules, ... to bracket roots
  \item We can find roots and reduce polynomials
  \item Newton's method is a simple local rule, but could be noisy
  \item Mullers method is a way to solve it more generally 
  \item Lots of methods available for special special cases
  \end{itemize}
\end{frame}

\end{document}

%%% Local Variables:
%%% mode: latex
%%% TeX-master: t
%%% End:
