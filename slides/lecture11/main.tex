\documentclass[10pt]{beamer}

\mode<presentation>
{
  \usetheme[height=1.25cm]{Madrid}
  \setbeamertemplate{navigation symbols}{}
  \setbeamercolor{alerted text}{fg=illini}
}

\graphicspath{{figs/}}

\usebackgroundtemplate{\includegraphics[width=\paperwidth,height=\paperheight]{uc-background}}

\usepackage[english]{babel}
\usepackage{epsfig,subfigure,bm}
\usepackage{multimedia}
\usepackage{psfrag}
\usepackage{animate}

\usefonttheme{metropolis} % default family is serif
%%%%%% Begin of my macros and options

\setbeamertemplate{section in toc shaded}[default][55]
\setbeamertemplate{subsection in toc shaded}[default][55]
\setbeamercolor{block title}{fg=white,bg=illini}
\setbeamercolor{block body}{fg=black,bg=mygrey}

\setbeamercolor{emphprimary}{fg=CBlue}
\setbeamercolor{emphsecondary}{fg=illini}
\setbeamercolor{emphtertiary}{fg=mygreen}
\definecolor{darkForestGreen}{rgb}{.1,1,.1}
\definecolor{veryLightGray}{rgb}{.9,.9,.9}
\definecolor{greenApple}{rgb}{.3,.9,.3}

\setbeamercolor{frametitle}{bg=CBlue}   
\setbeamercolor{title}{bg=CBlue}

\usepackage{amsmath,amssymb,amsxtra,amsthm}
\usepackage{algorithm,algorithmic}
\usepackage{natbib}
\usepackage{bibentry}
\usepackage{xspace}
\usepackage{changepage}

\definecolor{myblue}{rgb}{.2,.2,.7}
\definecolor{myred}{rgb}{.7,.2,.2}
\definecolor{mygreen}{rgb}{.2,.7,.2}
\definecolor{mygrey}{rgb}{0.9,0.9,0.9}
\definecolor{CBlue}{cmyk}{1,0.25,0,0}
\definecolor{illini}{rgb}{0.98,0.4,0.05}
\definecolor{black}{cmyk}{0,0,0,1}

\newcommand{\myemph}[1]{{\usebeamercolor[fg]{emphprimary}
    \textbf{#1}}}
\newcommand{\myemphalt}[1]{{\usebeamercolor[fg]{emphsecondary}
    \textbf{#1}}}

\graphicspath{{figs/}}

\title[Math for Robotics] % (optional, use only with long paper titles)
{CSE276C - Calculus of Variation}

\author[H.~I. Christensen] % (optional, use only with lots of authors)
{Henrik I.~Christensen}
% - Give the names in the same order as the appear in the paper.  -
% Use the \inst{?} command only if the authors have different
% affiliation.

\AtBeginSection[]
{
   \begin{frame}
       \frametitle{Outline}
       \tableofcontents[currentsection]
   \end{frame}
}

\institute[UCSD] % (optional, but mostly needed)
{
  \begin{minipage}[c]{.2\textwidth}
    \includegraphics[width=.65\linewidth]{ucsealnew}%
  \end{minipage}%
  \begin{minipage}[c]{.6\textwidth}
    \small
%%    \begin{center}
      Computer Science and Engineering\\
      University of California, San Diego\\
      \myemph{\url{http://cri.ucsd.edu}}\\          
%%    \end{center}

  \end{minipage}
%%  \vspace*{1ex}
}
%% - Use the \inst command only if there are several affiliations.
%% - Keep it simple, no one is interested in your street address.

\bigskip

\date[Nov 2020]% (optional, should be abbreviation of conference name)
{\small%
  November 2020}

\begin{document}
  
\nobibliography{/Users/hic/Dropbox/bibliography/bib-file}
\bibliographystyle{plain}

\begin{frame}[plain]
  \titlepage
\end{frame}


\begin{frame}
  \frametitle{Introduction}
  \begin{itemize}
  \item Going a bit more abstract today. 
  \item Calc of variations is tightly coupled to mechanics
  \item We will only covers the very basics 
  \item Entire courses at UCSD - MATH201C
  \end{itemize}
\end{frame}


\begin{frame}
  \frametitle{Applications}
  \begin{itemize}
  \item Path Optimization
  \item Vibrating membranes
  \item Electrostatics
  \item Machine vision - reconstruction
  \item VIsion - image flow, ...    
  \end{itemize}
\end{frame}

\begin{frame}
  \frametitle{Introduction (cont)}
  \begin{itemize}
  \item We have see the principle
    \begin{itemize}
    \item To minimize P is to solve P' = 0
    \end{itemize}
  \item So far we have looked at finite dimensional problems
    \begin{itemize}
    \item f: $\mathcal{R}^n \rightarrow \mathcal{R}$
    \end{itemize}
    Looking at N numbers to minimize f
  \item In infinite dimensional problems we are considering an continuum
  \item What about functionals - (functions of functions)? 
  \end{itemize}
\end{frame}

\begin{frame}
  \frametitle{Example}
  \begin{itemize}
  \item Suppose we connect two points in the plane $(x_0, y_0)$ and
    $(x_1, y_1)$ by a curve of the form y = y(x).
    \centerline{\includegraphics[height=4cm]{basic-curve}}
  \item The length if the curve can be written
    \[ L(y) = \int_{x_0}^{x_1} \sqrt{1 + (y')^2} dx \]
    L is a functional. 
  \item Find the shortest curve between the two points. 
  \end{itemize}
\end{frame}

\begin{frame}
  \frametitle{Similar problems}
  \begin{itemize}
  \item Shortest path connecting a non-planar curve, say sphere
  \item Minimal surface of revolution generated by a connected curve
  \item Shortest curve with a given area below it
  \item Closed curve of a given perimeter that encloses the largest area
  \item Shape of a string hanging from two points under gravity
  \item Path of light travelling through an inhomogenous curve
  \end{itemize}
\end{frame}


\begin{frame}
  \frametitle{Euler's Equation}
  \begin{itemize}
  \item The principle of 
    \begin{itemize}
    \item To minimize P is to solve P' = 0
    \end{itemize}
  \item Rather than solving the integral it is an advantage to
    consider the differential equation. 
  \item The differential equation is called Euler Equation. 
  \item We will derive it shortly
  \end{itemize}
\end{frame}


\begin{frame}
  \frametitle{Consider for a minute}
  \begin{itemize}
  \item Suppose $ f: \mathcal{R}^n \rightarrow \mathcal{R}$ what does it mean for $x^*$ to be a local extremum of f? 
    \begin{enumerate}
    \item We must have $f(x) \geq f(x^*)$ for every x in some neighborhood
    \item A necessary condition $\nabla f(x^*) = 0$ i.e., that $\frac{\partial f}{\partial x_i} = 0$ for all i. 
    \end{enumerate}
  \item For P the equivalent would be say
    \begin{enumerate}
    \item $P: C^2(\mathcal{R}^n) \rightarrow \mathcal{R}$ and
    \item $f \rightarrow P(f)$
    \end{enumerate}
  \item what does it mean for $f^*$ to be an extremum of P?
  \end{itemize}
\end{frame}

\begin{frame}
  \frametitle{Optimal functional? }
  \begin{itemize}
  \item What would be conditional for a functional?
    \begin{enumerate}
    \item We need $P(f) \geq P(f^*)$ for every functional close to $f^*$
      \begin{itemize}
      \item So what is a neighborhood of a function?
      \end{itemize}
    \item Need a generalized gradient
      \[
        P( f^* + \delta f ) \approx P(f^*)
      \]
    \end{enumerate}
  \item Still very hand wavy
  \end{itemize}
\end{frame}

\begin{frame}
  \frametitle{Simplest problem}
  \begin{itemize}
  \item Lets start with a simple problem
  \item Minimize $J(y) = \int_{x_0}^{x_1} F(x, y, y') dx$ with $y, F in C^2$
  \item Suppose $y^*$ minimizes J it would then be true
    \begin{enumerate}
    \item In a neighborhood of $y^*$ then $J(y) \geq J(y^*)$
    \item $\delta J = 0$ for a variation $\delta y$ is
      \[
        \delta J(y^*) = J(y^* + \delta y) - J(y^*)
      \]
    \end{enumerate}
  \item What are the necessary conditions for this to be valid
  \end{itemize}
\end{frame}

\begin{frame}
  \frametitle{Neighborhood Evaluation}
  \begin{itemize}
  \item Lets start by showing optimality in a neighborhood
  \item Let $y \in C^2[x_0, x_1]$ such that $y(x_0) = y(x_1) = 0$
  \item Let   $\epsilon \in \mathcal{R}$ be a value
  \item Lets consider a one-parameter family of functions
    \[
      y(x) = y^*(x) + \epsilon y(x)
    \]
  \item Where $y^*$ is the (unknown) optimal function
  \item Define $\Phi: \mathcal{R} \rightarrow \mathcal{R}$ by
    \[
      \Phi(\epsilon) = \int_{x_0}^{x_1} F(x, y, y') dx
    \]
  \item If $|\epsilon|$ is small enough then all variants of
    $y^* + \epsilon y$ lie in a small neighborhood of $y^*$,
    therefore $\Phi$ attains a local minimum at $\epsilon = 0$
  \item Thus it must be true that  $\Phi'(0) = 0$
  \end{itemize}
\end{frame}

\begin{frame}
  \frametitle{So what is $\Phi'$?}
  \begin{itemize}
  \item We know that
    \[
      \Phi(\epsilon) = \int_{x_0}^{x_1} F(x, y, y{'}) dx
    \]
  \item So it must be true that
    \[
      \Phi'(\epsilon) = \frac{d}{d \epsilon} \int_{x_0}^{x_1} F(x, y, y{'}) dx
    \]
  \item Given that we have a $C^{2}$ domain we can reverse the order of
    integration and differentiation, so that
    \[
      \Phi{'}(\epsilon) =  \int_{x_0}^{x_1} \frac{d}{d \epsilon} F(x, y, y{'}) dx
    \] \pause or
    \[
      \Phi{'}(\epsilon) =  \int_{x_0}^{x_1} \left( \frac{\partial}{\partial y} F(x, y^{*} + \epsilon y, y^{*}'+ \epsilon y' ) y + \frac{\partial}{\partial y'} F ( x, y^{*} + \epsilon y, y^{*} '+ \epsilon y{'} ) y{'} \right) dx
    \]
  \item We know that
    \[
      \Phi{'}(0) = 0 =   \int_{x_0}^{x_1} \left( \frac{\partial}{\partial y} F ( x, y^{*}, y^{*}' ) y + \frac{\partial }{\partial y'}  F( x, y^{*} , y^{*}{'} ) y{'} \right)  dx
    \]
  \end{itemize}
\end{frame}

\begin{frame}
  \frametitle{Still more $\Phi'$}
  \begin{itemize}
  \item We can write this more compactly
    \[
      \Phi{'}(0)= \int_{x_0}^{x_1} \left( F_y y + F_{y'} y' \right) dx
    \]
  \item Using integration by parts we get
    \[
      \begin{array}{rcl}
        \int_{x_0}^{x_1} F_{y'} y' dx &=& \left. F_{y'} y \right\vert_{x_0}^{x_1} - \int_{x_0}^{x_1} y \frac{d}{dx} F_{y'} dx\\
        &=& - \int_{x_0}^{x_1} y \frac{d}{dx} F_{y'} dx\\
      \end{array}
    \] with this we can rewrite
    \[
      \Phi{'}(0)= \int_{x_0}^{x_1} \left[ F_y - \frac{d}{dx} F_{y'} \right] y dx = 0
    \] as this has to apply for any function y it must be true that
    \[
      F_y - \frac{d}{dx} F_{y'} = 0 \mbox{ on } [x_0, x_1]
    \]
  \item This is called Euler's Equation
  \end{itemize}
\end{frame}



\begin{frame}
%   \frametitle{Summary}
% \end{frame}
\end{document}

%%% Local Variables:
%%% mode: latex
%%% TeX-master: t
%%% End:
