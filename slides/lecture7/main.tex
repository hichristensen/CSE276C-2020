\documentclass[10pt]{beamer}

\mode<presentation>
{
  \usetheme[height=1.25cm]{Madrid}
  \setbeamertemplate{navigation symbols}{}
  \setbeamercolor{alerted text}{fg=illini}
}
\usebackgroundtemplate{\includegraphics[width=\paperwidth,height=\paperheight]{uc-background}}

\usepackage[english]{babel}
\usepackage{epsfig,subfigure,bm}
\usepackage{multimedia}
\usepackage{psfrag}
\usepackage{animate}

%%%%%% Begin of my macros and options

\setbeamertemplate{section in toc shaded}[default][55]
\setbeamertemplate{subsection in toc shaded}[default][55]
\setbeamercolor{block title}{fg=white,bg=illini}
\setbeamercolor{block body}{fg=black,bg=mygrey}

\setbeamercolor{emphprimary}{fg=CBlue}
\setbeamercolor{emphsecondary}{fg=illini}
\setbeamercolor{emphtertiary}{fg=mygreen}
\definecolor{darkForestGreen}{rgb}{.1,1,.1}
\definecolor{veryLightGray}{rgb}{.9,.9,.9}
\definecolor{greenApple}{rgb}{.3,.9,.3}

\setbeamercolor{frametitle}{bg=CBlue}   
\setbeamercolor{title}{bg=CBlue}

\usepackage{amsmath,amssymb,amsxtra,amsthm}
\usepackage{algorithm,algorithmic}
\usepackage{natbib}
\usepackage{bibentry}
\usepackage{xspace}
\usepackage{changepage}

\definecolor{myblue}{rgb}{.2,.2,.7}
\definecolor{myred}{rgb}{.7,.2,.2}
\definecolor{mygreen}{rgb}{.2,.7,.2}
\definecolor{mygrey}{rgb}{0.9,0.9,0.9}
\definecolor{CBlue}{cmyk}{1,0.25,0,0}
\definecolor{illini}{rgb}{0.98,0.4,0.05}
\definecolor{black}{cmyk}{0,0,0,1}

\newcommand{\myemph}[1]{{\usebeamercolor[fg]{emphprimary}
    \textbf{#1}}}
\newcommand{\myemphalt}[1]{{\usebeamercolor[fg]{emphsecondary}
    \textbf{#1}}}

\graphicspath{{figs/}}

\title[Math for Robotics] % (optional, use only with long paper titles)
{CSE276C - Integration of Functions}

\author[H.~I. Christensen] % (optional, use only with lots of authors)
{Henrik I.~Christensen}
% - Give the names in the same order as the appear in the paper.  -
% Use the \inst{?} command only if the authors have different
% affiliation.

\AtBeginSection[]
{
   \begin{frame}
       \frametitle{Outline}
       \tableofcontents[currentsection]
   \end{frame}
}

\institute[UCSD] % (optional, but mostly needed)
{
  \begin{minipage}[c]{.2\textwidth}
    \includegraphics[width=.65\linewidth]{ucsealnew}%
  \end{minipage}%
  \begin{minipage}[c]{.6\textwidth}
    \small
%%    \begin{center}
      Computer Science and Engineering\\
      University of California, San Diego\\
      \myemph{\url{http://cri.ucsd.edu}}\\          
%%    \end{center}

  \end{minipage}
%%  \vspace*{1ex}
}
%% - Use the \inst command only if there are several affiliations.
%% - Keep it simple, no one is interested in your street address.

\bigskip

\date[Oct 2020]% (optional, should be abbreviation of conference name)
{\small%
  October 2020}

\begin{document}
  
\nobibliography{/Users/hic/Dropbox/bibliography/bib-file}
\bibliographystyle{plain}

\begin{frame}[plain]
  \titlepage
\end{frame}

\section{Introduction}

\begin{frame}
  \frametitle{Introduction}
  \begin{itemize}
  \item Interested in integration of function to allow estimation of
    future value
  \item Lots of potential applications in robotics
    \begin{itemize}
    \item Position estimation
    \item Path optimization
    \item Image restoration
    \end{itemize}
  \item Consider both end-point and boundary value problems, which
    anchors the problem
  \end{itemize}
\end{frame}

\begin{frame}
  \frametitle{Introduction - Setting the stage}
  \begin{itemize}
  \item We are trying to solve
    \[
      I = \int_a^b f(x) dx
    \]
    \item trying to solve I = y(b) for the equation
    \[
      \frac{\partial y}{\partial x} = f(x)
    \] 
  \item with the boundary condition
    \[
      y(a) = 0
    \]
  \item Objective to generate a good estimate of y(b) with a
    reasonable number of evaluations
  \item Emphasis on 1D problems, but in most cases generalization is
    straight forward
  \end{itemize}
\end{frame}

\begin{frame}
  \frametitle{Basic use of simpsons rule}
  \begin{itemize}
  \item Consider equally spaces data points
    \[
      x_i = x_0 + i h \mbox{ ~~~~~~~~} i = 0, 1, ..., N
    \]
    
  \item the function is evaluated at $x_i$
    \[
      f_i = f(x_i)
    \]
  \item The Newton-Cotes rules is then
    \[
      \int_{x_0}^{x_1} f(x) dx = \frac{f_1-f_0}{2} h + O(f'' h^3)
    \]
  \item The Simpson rules is
    \[
      \int_{x_0}^{x_2} f(x) dx = \frac{h}{3} (f_0 + 4 f_1 + f_2) + O(h^5 f^{(4)})
    \]
  \item which is exact to the 3rd degree
  \item The simpson $\frac{3}{8}$ rule
    \[
      \int_{x_0}^{x_3} f(x) dx = \frac{h}{8} (3 f_0 + 9 f_1 + 9 f_2 + 3 f_3)
    \]
  \item There are a series of rules for higher order, check literature
  \end{itemize}
\end{frame}

\begin{frame}
  \frametitle{Simpson / Trapezoid Rules}
  \begin{itemize}
  \item Clearly the local rules can be chained into a longer evaluation
  \item $(x_0, x_1), (x_1, x_2), \ldots, (x_{N-1},x_N)$ to get an extended
    trapezoid form
    \[
      \int_{x_0}^{x_N} f(x) dx = h(\frac{1}{2} f_0 + f_1 + f_2 + \ldots + f_{N-1} + \frac{1}{2} f_N )
    \]
  \item The error estimate is
    \[
      O\left( \frac{(x_N - x_0) f''}{N^2} \right)
    \]
  \end{itemize}
\end{frame}

\begin{frame}
  \frametitle{Trapezoid Rule - Strategy?}
  \begin{itemize}
  \item How can you effective use the trapezoid rule?
    \pause
  \item Use of a coarse to fine strategy and watch convergence
  \item This is termed Romberg integration in numerical toolboxes
  \item In general these methods generate good accuracy for proper functions? 
  \end{itemize}
\end{frame}

\begin{frame}
  \frametitle{Handling of improper function}
  \begin{itemize}
  \item What is an improper function?
    \pause
    \begin{enumerate}
    \item Integrand goes to a finite value but cannot be evaluated at a point, such as
      \[
        \frac{sin x}{x} \mbox{ ~~~ at ~~~ } x=0
      \]
    \item Upper limit is $\infty$ or lower limit is $-\infty$
    \item Has a singularity at a boundary point, e.g.,
      \[
        x^{-1/2} \mbox{~~~~ at ~~~~~} x = 0
      \]
    \item Has a singularity with the interval at a known location
    \item If the value is infinite, e.g.,
      \[
        \int_0^{\infty} x_{-1} dx \mbox{~~~~or~~~~} \int_{-infty}^{\infty} cos x dx
      \] it is not improper but impossible
    \end{enumerate}
  \end{itemize}
\end{frame}


\begin{frame}
  \frametitle{Questions}
  \centerline{\Huge Questions}
\end{frame}

\end{document}

%%% Local Variables:
%%% mode: latex
%%% TeX-master: t
%%% End:
